\documentclass[12pt,a4paper]{article}

\usepackage[francais]{babel}
\usepackage[utf8]{inputenc}
\usepackage{fullpage}
\usepackage{amssymb}
\usepackage{hyperref}

\title{Machine virtuelle de l'université de Marne-la-Vallée}
\author{Cours de compilation L3}

% %%%%%%%%%%%%%%%%%%%%%%%%%%%%%%%%%%%%%%%%
% 
% This file is part of the documentation of virtual_mlv.
% 
% virtual_mlv is free software: you can redistribute it and/or modify
% it under the terms of the GNU General Public License as published
% by the Free Software Foundation, either version 3 of the License,
% or (at your option) any later version.
% 
% virtual_mlv is distributed in the hope that it will be useful, but
% WITHOUT ANY WARRANTY; without even the implied warranty of
% MERCHANTABILITY or FITNESS FOR A PARTICULAR PURPOSE.  See the GNU
% General Public License for more details.
% 
% You should have received a copy of the GNU General Public License
% along with virtual_mlv.  If not, see
% <http://www.gnu.org/licenses/>.
% 
% 
% Authors: S. Lombardy, N. Bedon, C. Morvan, W. Hay
% %%%%%%%%%%%%%%%%%%%%%%%%%%%%%%%%%%%%%%%%


\begin{document}
\maketitle

\bigskip
\section{Introduction}

Dans le cadre du cours de compilation du L3 informatique de
l'université de Marne-la-Vallée, nous utiliserons une {\em
  machine virtuelle}. Il s'agit d'une machine abstraite à pile.  Elle
est composée de:
\begin{itemize}
\item un segment de code;
\item une pile de données;
\item deux registres de travail {\tt reg1} et {\tt reg2};
\item un registre {\tt base} qui pointe sur un emplacement
  particulier de la pile (il est manipulé uniquement par la machine
  lors des appels de fonction).
\item un registre {\tt counter} qui pointe sur un emplacement
  particulier du segment de code, c'est le compteur ordinal de la
  machine.
\end{itemize}

Le segment de code contient la représentation du {\it v-code} exécuté
par la machine. Cette partie de la mémoire est statique et reste
inchangée durant l'exécution de la machine virtuelle. L'adresse du
début de ce segment est {\tt 0}. Ce segment de code peut être vu comme
un tableau, chaque case contenant un entier représentant soit une
instruction, soit l'argument d'une instruction.

La pile de données permet de stocker toutes les valeurs nécessaires à
l'exécution du programme: variables globales, locales et temporaires,
mais elle sert aussi de pile d'exécution: lors d'un appel de fonction,
on y stocke l'état de la machine virtuelle.

Les deux registres de travail sont ceux sur lesquels s'effectuent
toutes les opérations. Le registre {\tt base} permet d'indiquer la
portion de la pile qui correspond à l'exécution de la fonction
courante.

\section{Langage de la machine virtuelle}
Le langage de la machine virtuelle est très simple. Chaque ligne
contient une commande et, éventuellement un
argument (entier) pour cette commande.
Les commentaires sont introduits par {\tt \#} et durent jusqu'à la
fin de la ligne courante.

Les commandes sans argument acceptées par la machine virtuelle sont
représentés sur la Figure~\ref{fig:instruct}, celles possédant un
argument, sur la Figure~\ref{fig:instructarg}

\begin{figure}
  \centering
  \begin{tabular}{lll}
    NOP   &:& Ne fait rien du tout\\
    NEG   &:& {\tt reg1}$\leftarrow-${\tt reg1}\\
    ADD   &:& {\tt reg1}$\leftarrow${\tt reg1}$+${\tt reg2}\\
    SUB   &:& {\tt reg1}$\leftarrow${\tt reg1}$-${\tt reg2}\\
    MUL   &:& {\tt reg1}$\leftarrow${\tt reg1}$*${\tt reg2}\\
    DIV   &:& {\tt reg1}$\leftarrow${\tt reg1}$/${\tt reg2}\\
    MOD   &:& {\tt reg1}$\leftarrow${\tt reg1}$\%${\tt reg2}\\
    EQUAL &:& {\tt reg1}$\leftarrow${\tt reg1}$=${\tt reg2}\\
    NOTEQ &:& {\tt reg1}$\leftarrow${\tt reg1}$\neq${\tt reg2}\\
    LESS  &:& {\tt reg1}$\leftarrow${\tt reg1}$<${\tt reg2}\\
    LEQ   &:& {\tt reg1}$\leftarrow${\tt reg1}$\leqslant${\tt reg2}\\
    GREATER &:& {\tt reg1}$\leftarrow${\tt reg1}$>${\tt reg2}\\
    GEQ   &:& {\tt reg1}$\leftarrow${\tt reg1}$\geqslant${\tt reg2}\\
    PUSH  &:& Place la valeur de {\tt reg1} sur la pile\\
    POP   &:& Place la valeur en tête de pile dans {\tt reg1} et dépile\\
    SWAP  &:& Échange les valeurs de {\tt reg1} et {\tt reg2}\\
    READ  &:& Lit un entier et le stocke en {\tt reg1}\\
    READCH  &:& Lit un caractère et le stocke en {\tt reg1} (comme entier)\\
    WRITE &:& Affiche la valeur stockée en {\tt reg1}\\
    WRITECH &:& Affiche le contenu de {\tt reg1} vu comme un caractère\\
    HALT  &:& Arrête l'exécution de la machine\\
    RETURN&:& Termine l'activation de la fonction en cours
    et retourne à la\\&& fonction appelante (voir plus loin)\\
    LOAD &:& Place dans {\tt reg1} la valeur située à l'adresse {\tt reg1}
    de la pile\\
    LOADR &:& Place dans {\tt reg1} la valeur située à l'adresse
    {\tt reg1}$+${\tt base} de la pile\\
    SAVE &:& Stocke la valeur de {\tt reg1} à l'adresse {\tt reg2} de la pile\\
    SAVER&:& Stocke la valeur de {\tt reg1} à l'adresse {\tt base}$+${\tt reg2} de la pile\\
    TOPST &:& Place dans {\tt reg1} la position courante du sommet de la pile
  \end{tabular}
  
  \caption{Instructions de la machine virtuelle}
  \label{fig:instruct}
\end{figure}

\begin{figure}
  \centering
  \begin{tabular}{llll}
    SET   &$n$&:& {\tt reg1}$\leftarrow n$\\
    LABEL &$n$&:& Déclare le label numéro $n$\\
    JUMP  &$n$&:& Branchement à l'emplacement du label $n$ dans le segment de code\\
    JUMPF &$n$&:& Branchement à l'emplacement du label $n$ dans le segment de code\\
    &&& si {\tt reg1} vaut $0$\\
    ALLOC &$n$&:& Alloue $n$ emplacements supplémentaires en tête de
    pile\\
    FREE  &$n$&:& Libère $n$ emplacements en tête de pile\\
    CALL  &$n$&:& Sauvegarde l'état de la machine dans la pile et effectue \\
    &&& un branchement à l'emplacement du label $n$ dans le segment de code\\
  \end{tabular}
  \caption{Instructions avec argument}
\label{fig:instructarg}
\end{figure}

Après chaque instruction, on passe à l'instruction suivante
(incrémentation du registre {\tt counter}), excepté
pour les instructions {\tt JUMP}, {\tt JUMPF} et {\tt CALL} qui
induisent explicitement un branchement dans le segment de code
et l'instruction {\tt RETURN} qui provoque un branchement dépendant
des instructions stockées lors de l'activation de la fonction.

Notez que lors du chargement du programme par la machine virtuelle,
les labels sont effacés et les branchements se font selon l'adresse
des instructions dans le segment de code.

L'appel de {\tt CALL} provoque l'empilement successif d'un pointeur
indiquant quelle instruction exécuter au
retour de la fonction, et de {\tt base}. À la suite de quoi,
{\tt base} est mis à jour pour pointer sur le sommet de la pile
d'exécution
qui correspondra au début de la zone dédiée à la fonction appelée.
Attention, les registres ne sont pas automatiquement sauvegardés lors
de l'appel d'une fonction.

{\tt RETURN} supprime la portion de pile dédiée à la fonction
courante, dépile les valeurs empilées lors de l'appel de la fonction
et restaure {\tt base} ainsi que le pointeur sur le segment de code.

\section{Appel de fonction}
L'appel de fonction ne gère pas les paramètres, et la commande
{\tt RETURN} ne gère pas la valeur de retour.  Pour appeler une
fonction avec $n$ arguments, on pourra empiler ces $n$ arguments sur
la pile d'exécution puis procéder à l'appel de la fonction. Leur
adresse relative sera alors située entre $-n-2$ et $-3$, les
emplacements d'adresse respective $-2$ et $-1$ étant occupés par les
informations nécessaires à la restauration de {\tt counter} et
{\tt base} lors du retour de la fonction.  La valeur de retour pourra
elle être placée dans {\tt reg1}. Une autre option peut être de lui
réserver, par convention le premier espace de la pile.

%\newpage
\section{Exemple}
Exemple de code accepté par la machine virtuelle:
\begin{verbatim}
#      Commande Arg
        SET     4       #reg1 := 4
        PUSH            #push 4
        SET     8       #reg1 := 8
        SWAP            #reg2 := 8
        POP             #reg1 := 4
        LESS            #reg1 := 4<8
        JUMPF   0       #goto 0 si reg1=0
        SET     3       #reg1 := 3
        PUSH            #push 3
        SET     5       #reg1 := 5
        SWAP            #reg2 := 5
        POP             #reg1 := 3
        LESS            #reg1 := 3<5
        JUMPF   1       #goto 1 si reg1=0
        SET     4       #reg1 := 4
        PUSH            #push 4
        SET     70      #reg1 := 70
        SWAP            #reg2 := 70
        POP             #reg1 := 4
        ADD             #reg1 := 4+70
        WRITE           #---affichage 74
        LABEL   0
        LABEL   1
        SET     12      #reg1 := 12
        PUSH            #push 12
        SET     5       #reg1 := 5
        SWAP            #reg2 := 5
        POP             #reg1 := 12 
        LESS            #reg1 := 12<5
        JUMPF   2       #goto 2 si reg1=0
        SET     97
        WRITE           #---affichage 97
        WRITECH         #---affichage a
        LABEL   2
        HALT
\end{verbatim}

\section{Licence}

Cette documentation est fournie avec la machine virtuelle
\texttt{virtual\_mlv}. La machine virtuelle, comme la documentation
est placée sous license GPL, v3. Les auteurs en sont: Nicolas Bedon,
Quentin Campos, Gaël Fuhs, Wuthy Hay, Sylvain Lombardy, Jefferson
Mangue et Christophe Morvan.

Le code est disponible à l'adresse suivante :
\texttt{https://github.com/cmorvan/virtual\_mlv}

\end{document}


